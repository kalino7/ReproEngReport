% VLDB template version of 2020-08-03 enhances the ACM template, version 1.7.0:
% https://www.acm.org/publications/proceedings-template
% The ACM Latex guide provides further information about the ACM template

\documentclass[sigconf, nonacm]{acmart}

%% The following content must be adapted for the final version
% paper-specific
\newcommand\vldbdoi{XX.XX/XXX.XX}
\newcommand\vldbpages{XXX-XXX}
% issue-specific
\newcommand\vldbvolume{14}
\newcommand\vldbissue{1}
\newcommand\vldbyear{2020}
% should be fine as it is
\newcommand\vldbauthors{\authors}
\newcommand\vldbtitle{\shorttitle} 
% leave empty if no availability url should be set
\newcommand\vldbavailabilityurl{https://github.com/kalino7/ReproEngProject}
% whether page numbers should be shown or not, use 'plain' for review versions, 'empty' for camera ready
\newcommand\vldbpagestyle{plain} 

\begin{document}
\title{RepEng Project: An Approach for Schema Extraction of JSON and Extended JSON Document Collections}

%%
%% The "author" command and its associated commands are used to define the authors and their affiliations.
\author{Kachimsirikwuo Caleb Imo}
\affiliation{%
  \institution{Universität Passau}
  \streetaddress{P.O. Box 1212}
  \city{Passau}
  \state{Germany}
}
\email{imo01@ads.uni-passau.de}


\maketitle

%%% do not modify the following VLDB block %%
%%% VLDB block start %%%
\ifdefempty{\vldbavailabilityurl}{}{
\vspace{.3cm}
\begingroup\small\noindent\raggedright\textbf{Artifact Availability:}\\
The source code, data, and/or other artifacts have been made available at \url{\vldbavailabilityurl}.
\endgroup
}
%%% VLDB block end %%%
\section{Introduction}
This project report aims to reproduce the findings presented in the original paper \cite{frozza2018approach}. The original study introduced an innovative approach to extracting JSON or Extended JSON schema from large datasets.

The introduction of NoSQL databases has revolutionized data storage by enabling the storage of diverse data collections without requiring a predefined structure. However, the absence of an explicitly stated schema does not imply the absence of an implicit schema \cite{sevilla2015inferring} \cite{frozza2018approach}. The proposed approach in \cite{frozza2018approach}, termed JSON Schema Discovery, aims to extract a unified schema from a collection of JSON documents, catering to applications that may necessitate such schema information.

The JSON Schema Discovery approach has four phases: \textit{(i) raw schema generation}: which generates the schema from the documents and store in a database (mongoDB); \textit{(ii) Grouping of generated schema}: The generated schemas are reordered in alphabetical order, facilitating the removal of possible duplicates; \textit{(iii) Unification of raw schemas } and \textit{(iv) JSON Schema generation} \cite{frozza2018approach}.

\section{TOOL EVALUATION}
To evaluate the quality of the JSON Schema Discovery approach, three experiments were conducted using diverse collections of JSON documents stored in a MongoDB. 
Table \ref{foursquare} shows the average processing times for schema extraction from datasets related to venues, check-ins, and tweets obtained from Foursquare \cite{ccelikten2016modeling} \cite{frozza2018approach}. The following terms are derived from table \ref{foursquare}:

\begin{table}
\centering
\caption{Results for Foursquare Datasets \cite{frozza2018approach}}\label{foursquare}
\scalebox{0.79} {
\begin{tabular}{|l|c|c|c|c|c|c|}
\hline
\textbf{Collection} & \textbf{N\_JSON} & \textbf{RS} & \textbf{ROrd} & \textbf{TB} & \textbf{TT} & \textbf{TB/TT} \\
\hline
venues & 2 million & 257 & 117 & 7.47 min & 7.52 min & 99.33\% \\
\hline 
checkins & 11 million & 2 & 2 & 35.27 min & 35.52 min & 99.29\% \\
\hline
tweets & 17 million & 23 & 16 & 53.11 min & 53.44 min & 99.38\% \\
\hline
\end{tabular}
}
\end{table}

\begin{description}
  \item[\textbullet] \textbf{N\_JSON}: denotes the number of JSON documents/records.
  \item[\textbullet] \textbf{RS}: denotes the number of distinct raw schemas that were generated.
  \item[\textbullet]\textbf{ROrd}: denotes the number of raw schemas generated with ordered structure
  \item[\textbullet]\textbf{TB}: denotes the time taken to generate the raw schemas
  \item[\textbullet]\textbf{TT}: denotes the total time taken to execute the entire process
\end{description}

\subsection{Research Question}
Given identical datasets, does running the experiment, using the same methodology outlined in the original research paper result in similar outcomes, thereby supporting the effectiveness and quality of the reported findings as shown in Table \ref{foursquare}?

The experiment in the original research paper, as depicted in Table \cite{ccelikten2016modeling}, were conducted on an Amazon EC2 t2.micro instance (Intel® Xeon® E5-2676 v3 @ 2.40GHz and 1GB of RAM) \cite{frozza2018approach}. In contrast, the reproduction of this experiment will be conducted on an ACER Aspire A515-55 system ( Intel(R) Core(TM) i5-1035G1 CPU @ 1.00GHz).

To validate the outcomes of this reproduction paper, I will measure and compare, against the original paper, the processing time needed for raw schema extraction and the overall processing time for generating an output. Additionally, a thorough comparison will be conducted for the output values of both distinct and ordered generated schemas, with those reported in the original paper. This comparison aims to confirm the consistency and reliability of the results obtained in the original study.

This reproduction paper is exclusively dedicated to validating the experimental results presented in Table \ref{foursquare} and therefore will not undertake the reproduction of any other experiments conducted in \cite{frozza2018approach}. The focused scope of this study ensures a thorough and meticulous examination of the specific findings outlined in the referenced table.

\subsection{Factors To Consider}
Considering the time gap between the original study and this reproduction paper, potential challenges may arise in retrieving the actual datasets used in the original experiment. Datasets could have been modified, completely changed, or even erased. These factors will be duly acknowledged and addressed in subsequent sections of this paper.

\bibliographystyle{ACM-Reference-Format}
\bibliography{sample}

\end{document}
\endinput
